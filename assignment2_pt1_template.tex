% ====================================================================
%  Machine Learning Assignment Template
%  Title  : Shakespeare‑specific Chatbot – Prompt Engineering & Design
%  Authors : <Student Name(s) – Student ID(s)>
%  Date   : \today
% --------------------------------------------------------------------
%  Notes for students
%  • Length: 1,500 – 2,000 words
%  • Replace the placeholders surrounded by <angle brackets>.
%  • Keep all \section titles unchanged so that marking rubrics align.
%  • Use the custom boxes below to present your prompt / response pairs.
%  • Cite only peer‑reviewed literature or official tool documentation.
% ====================================================================

\documentclass[11pt,a4paper]{article}

% ---------- Packages ----------
\usepackage[margin=2.5cm]{geometry}
\usepackage{setspace}           % line spacing
\usepackage{graphicx}           % figures
\usepackage{hyperref}           % hyperlinks
\usepackage[numbers]{natbib}    % bibliography
\usepackage{caption}            % caption formatting
\usepackage{booktabs}           % nicer tables
\usepackage[most]{tcolorbox}          % prompt / response boxes
\usepackage{parskip}            % no paragraph indent
\usepackage{xcolor}             % colour definitions
\usepackage{enumitem}           % tight itemise / enumerate


% ---------- Box styles ----------
\tcbset{
	enhanced,
	sharp corners,
	boxrule=0.4pt,
	left=2mm,right=2mm,top=1mm,bottom=1mm
}

\newtcolorbox{promptbox}{
	colback=orange!10,
	colframe=orange!60!black,
	title=\textbf{Student Prompt},
}

\newtcolorbox{responsebox}{
	colback=blue!5,
	colframe=blue!60!black,
	title=\textbf{ChatGPT Response},
}

\newtcolorbox{insightbox}{
	colback=green!8,
	colframe=green!50!black,
	title=\textbf{Insight / Reflection},
}

% ---------- Metadata ----------
\title{Developing a Shakespeare‑Specific Chatbot\\
	\large Prompt Engineering, Validation, and System Design}
\author{Full Name of students in group (Student ID)\\
	CSCI433/933\\
	University of Wollongong, Australia}
\date{\today}

% ====================================================================
\begin{document}
	\maketitle
	\section{Introduction}
	Briefly describe the project objective, scope (150–200 words), and how
	LLMs assist your investigation.
	
	% --------------------------------------------------------------------
	\section{Prompt Engineering Workflow}
	\subsection{Strategy}
	Explain your overall prompting strategy and how you iteratively refined
	questions to drill down into model size, data pipelines, and deployment
	constraints.
	
	\subsection{Prompt / Response Log}
	For each interaction that materially changed your understanding,
	include:
	
	% ---------- Example interaction ----------
	\begin{promptbox}
		How can I fine‑tune a distilled transformer on a domain‑specific corpus
		containing Early Modern English (Shakespeare) with limited GPU memory?
	\end{promptbox}
	
	\begin{responsebox}
		ChatGPT (excerpt):  
		You can start with DistilGPT‑2 and use parameter‑efficient tuning
		such as LoRA… <trimmed for brevity>
	\end{responsebox}
	
	\begin{insightbox}
		\textbf{Action taken:} Switched search focus to LoRA + 8‑bit
		quantisation strategies; formulated next prompt on LoRA rank choice.
	\end{insightbox}
	% ---------- /Example ----------
	
	Repeat the triad (prompt / response / insight) as needed.  
	\textbf{Important:} redact any irrelevant conversational fluff.
	
	% --------------------------------------------------------------------
	\section{Critical Appraisal of ChatGPT Outputs}
	Assess accuracy and completeness of the model’s answers. Organise by
	theme (model architecture, data preparation, generation tasks, tools).
	Reference literature to support or contest specific claims.
	
	% --------------------------------------------------------------------
	\section{Literature Cross‑Validation}
	\begin{enumerate}[label=\arabic*.]
		\item \textbf{Model Architectures:} summarise 2–3 peer‑reviewed
		papers on lightweight transformers or RNNs for niche domains.
		\item \textbf{Data Preparation:} cite studies on cleaning Shakespearean
		or Early Modern English corpora.
		\item \textbf{Generation and Summarisation:} discuss methods for
		controllable text generation and scene‑based summarisation.
		\item \textbf{Deployment Tools:} reference official documentation or
		conference papers on HuggingFace, LangChain, RAG pipelines, etc.
	\end{enumerate}
	
	% --------------------------------------------------------------------
	\section{Proposed System Design (Part Two)}
	Present an end‑to‑end architecture diagram 
	%(Figure \ref{fig:architecture})
	and explain each component: data ingestion, fine‑tuning, retrieval
	augmentation, inference API, UI. Highlight resource constraints and how
	choices were informed by ChatGPT dialogue and literature.
	
	\begin{figure}[h]
		\centering
	%	\includegraphics[width=0.9\textwidth]{architecture_placeholder}
		\caption{High‑level system design for a Shakespeare‑specific chatbot.}
		\label{fig:architecture}
	\end{figure}
	
	% --------------------------------------------------------------------
	\section{Conclusion}
	Summarise key takeaways about prompt‑driven inquiry, insights gained,
	and next steps.
	
	% --------------------------------------------------------------------
%	\section*{Word Count}
%	% Update the count manually or with \texttt{texcount} if required.
%	\textbf{Approximate narrative word count:} $<XXXX>$ (target 1 500–2 000).
%	
	% --------------------------------------------------------------------
	\bibliographystyle{plainnat}
	\bibliography{references}   % Create references.bib in the same folder
	
	% --------------------------------------------------------------------
%	\appendix
%	\section{Complete Prompt / Response Transcript (Optional)}
%	Include full, unabridged dialogue here if required, using the same box
%	format.
	
\end{document}
